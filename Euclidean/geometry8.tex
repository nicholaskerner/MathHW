%Latex2e file
\documentclass[12pt,letterpaper]{article}
%\renewcommand{\arraystretch}{2}
%\input{\scrload.tex}
\setlength{\textwidth}{6.5in}
\setlength{\textheight}{8.6in}

\setlength{\oddsidemargin}{-.25in}
\setlength{\evensidemargin}{-.25in}
\setlength{\topmargin}{-.25in}
\pagestyle{empty}

\usepackage{amsmath}
\usepackage{amssymb}
\usepackage{graphicx}

\newcommand{\R}{\ensuremath{{\mathbb{R}}}}
\newcommand{\Z}{\ensuremath{{\mathbb{Z}}}}
\newcommand{\Q}{\ensuremath{{\mathbb{Q}}}}
\newcommand{\N}{\ensuremath{{\mathbb{N}}}}
\newcommand{\C}{\ensuremath{{\mathbb{C}}}}
\newcommand{\Proof}{\noindent {\bf Proof: }}
\newcommand{\QED}{\begin{flushright}QED\end{flushright}}
\newcommand{\Refl}{{\bf Reflexive: }}
\newcommand{\Symm}{{\bf Symmetric: }}
\newcommand{\Tran}{{\bf Transitive: }}
\newcommand{\ep}{\varepsilon}
\newcommand{\ri}{\right|}
\newcommand{\lef}{\left|}
\newcommand{\toR}{\to \R}
\newcommand{\fancy}[1]{#1_{\text{fancy}}}
\newcommand{\pro}[1]{\noindent {\bf #1}}
\newcommand{\prob}[1]{\newpage\noindent {\bf #1}}
\newcommand{\bacon}{\approx}

   
\begin{document}
\begin{flushright}
Nick Kerner

Homework 8

Chapter 6: 9

Dr Frey's Bonus but required fun question

Chapter 7: K11, K20

\end{flushright}
\begin{center}
\large{Geometry}\\
\end{center}

\pro{Chapter 6: problem 9 }

Given $X\neq P $ on l, by proposition 6.6 we know there are two limiting parallel rays to l emanating from P.  Choose the one on the same side of $\overleftrightarrow{PQ}$ as X and drop a perpendicular from X to our limiting parallel ray and call the foot Y.   Know we know that $\overrightarrow{PY}$ is interior to $\angle QPX$ ($\overrightarrow{PY}\neq \overrightarrow{PX}$ as limiting parallel rays cannot have a common perpendicular to their respective line, so XY is a segment not a point). \\

Case 1: $\overrightarrow{PY}$ is interior to $\angle XPX'$\\

Then by crossbar theorem we know that $\overrightarrow{PY}$ must intersect $XX'$. \\

Case 2: $\overrightarrow{PY} = \overrightarrow{PX'}$\\

Our limiting parallel cannot intersect l.  Contradiction.\\

Case 3: $\overrightarrow{PY}$ is interior to $\angle QPX'$.\\

By crossbar theorem, $\overrightarrow{PY}$ intersects $QX'$, but our limiting parallel cannot intersect l.  Contradiction.\\

Therefore we know that $\overrightarrow{PY}$ intersects $XX'$ (not at an endpoint as it would be the parallel with common perpendicular to $l'$ or it would intersect $l'$) call this point Z.  \\

Case 1: Y = Z.\\

$XY = XZ$, clearly. \\

Case 2: $Y\neq Z$.\\

Consider $\triangle XYZ$.  We know that $\angle XYZ$ is a right angle, so its measure is $90^\circ$, so since we are in hyperbolic a triangle's angle sum must be less than 180, so both other angles have measures less than $90^\circ$.  So we know by proposition 4.5 that since $\angle XYZ > \angle XZY$ then $XZ > XY$.\\

Hence $XZ \geq XY$.\\



By Aristotle's axiom, given any segment AB, there is some X with 



\prob{Dr Frey's Bonus but required fun question }

\prob{Chapter 7: K11 }

\prob{Chapter 7: K20 }



\end{document}