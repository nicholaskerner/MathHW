%Latex2e file
\documentclass[12pt,letterpaper]{article}
%\renewcommand{\arraystretch}{2}
%\input{\scrload.tex}
\setlength{\textwidth}{6.5in}
\setlength{\textheight}{8.5in}



\setlength{\oddsidemargin}{-.25in}
\setlength{\evensidemargin}{-.25in}
\setlength{\topmargin}{-.25in}
\pagestyle{empty}

\usepackage{amsmath}
\usepackage{amssymb}
\usepackage{graphicx}

\newcommand{\R}{\ensuremath{{\mathbb{R}}}}
\newcommand{\Z}{\ensuremath{{\mathbb{Z}}}}
\newcommand{\Q}{\ensuremath{{\mathbb{Q}}}}
\newcommand{\N}{\ensuremath{{\mathbb{N}}}}
\newcommand{\C}{\ensuremath{{\mathbb{C}}}}
\newcommand{\Proof}{\noindent {\bf Proof: }}
\newcommand{\QED}{\begin{flushright}QED\end{flushright}}
\newcommand{\Refl}{{\bf Reflexive: }}
\newcommand{\Symm}{{\bf Symmetric: }}
\newcommand{\Tran}{{\bf Transitive: }}
\newcommand{\ep}{\varepsilon}
\newcommand{\ri}{\right|}
\newcommand{\lef}{\left|}
\newcommand{\toR}{\to \R}
\newcommand{\fancy}[1]{#1_{\text{fancy}}}
\newcommand{\pro}[1]{\noindent {\bf #1}}
\newcommand{\prob}[1]{\newpage\noindent {\bf #1}}
\newcommand{\bacon}{\approx}

   
\begin{document}
\begin{flushright}
Nick Kerner

Homework 3

Chapter 2: Dr Frey's Bonus, 14, Major Exercise 2,3\\
Chapter 3: 14,15,16,17
\end{flushright}
\begin{center}
\large{Geometry}\\
\end{center}


\pro{Dr Frey's Bonus but required fun question}

Let M be a projective plane.  Define a new interpretation M' by taking as ``point'' of M' the lines of M and as ``lines'' of M' the points of M, with the same incidence relation.  Prove that M' is also a projective plane (called the dual plane of M).\\

First we must show that the axioms of incidence geometry are satisfied, then we must show that this model has the elliptic parallel property.  \\

I1: Given two distinct points in M', we know these correspond to two distinct lines in M and we know that they must meet since M has the elliptic parallel property.  Therefore let them meet at P. Suppose that they also meet at some other point.  Then by I1 in M, they would be the same line (contradiction).  Therefore they only meet at P.  In M', P is therefore a unique line incident with both given points in M'.  Therefore we have shown that any two points in M' are incident with a unique line.\\

Strengthened I2: Given a line P in M', we know this is a point in M.  We also know by I3 that there are 3 non-collinear points in M.  Call the other two points A and B, so A,B, and P are non-collinear.  However by I1 we know that a line containing A and B exists (call it AB), similarly, lines AP and BP exist.  Since M is a projective plan we know that each of these lines has a third point (strong I2).  Since AB must have a third point and since A,B, and P are non-collinear, we have discovered another point in the model, call it C. So ABC is our extended line.  Additionally, AP must have another point.  If we let it be B or C, then APB/APC would have two points in common with ABC, so by I1 they would be the same line which means that A, B and P are collinear (contradiction).  So we know there is another point, D, that is on AP (now APD).  Similarly BP must have a third point.  If we choose A then A,B, and P would be collinear (contradiction).  If we chose C BPC would be the same as ABC by I1, which would mean that A,B, and P would be collinear (contradiction).  If we chose D, APD would be the same as BPD, so A,B, and P would be collinear (contradiction).  Therefore we need another point.  Call this new point E, so we have BPE.\\

Now we also know that there must be a line incident with both P and C (I1), call it PC.  We already know that APD and BPE are distinct (if they were the same, A,B, and P would be collinear which would be a contradiction), so if we can show that PC is distinct from these other two lines, we will have shown that there are three lines incident with P. \\

Suppose to the contrary that PC is the same as APD.  Then by I1 we know that ABC is the same as APDC (as both A and C are in common).  This means that A,B and P are collinear.  Contradiction.\\

Suppose to the contrary that PC is the same as BPE.  Then PBEC is the same as ABC by I1, so A,B and P are collinear.  Contradiction. \\

Therefore PC, APD, and BPE are distinct lines.  Also, in M', they are points incident with the line P.  Therefore there are three distinct points on line P so strong I2 is satisfied.\\


I3: To show there are 3 non collinear points in M', we must show that there are 3 distinct lines in M that have no point mutually in common (obviously with the elliptic parallel property, each pair of lines has a point in common).  \\

By I3 we know there are 3 non-collinear points in M, let them be A,B, and C.  Then we have lines AB, BC, and AC.  If any of these lines are not distinct, then A,B, and C would be collinear which would be a contradiction.  Therefore we have three distinct lines.  Given two of these lines we know they have 1 point in common (by I1).  Suppose that point was in common with the third line.  Since we know the third point already has a point in common with each other line, it would have 2 points in common with each of the other lines.  Therefore all three lines would be the same line (I1), so A,B, and C would be collinear. Contradiction.  Therefore the three lines have no point mutually in common. \\

Therefore we know that we have 3 distinct points in M' (these three lines) which cannot be collinear as being collinear in M' means they have a point mutually in common in M.\\



Elliptic parallel property:  Given two lines in M', they are points in M.  Additionally, by I1, we know that there is a line connecting these two points in M.  This line is the point in M' at which they intersect.  Therefore there are no parallel lines in M'.  Thus, there are no parallel lines and M' has the elliptic parallel property.




\prob{14}  

a. Let S be the following statement in the language of incidence geometry:  If l and m are any two distinct lines, then there exists a point P that does not lie on either l or m.  Show that S is not a theorem in incidence geometry, i.e., cannot be proved from the axioms of incidence geometry.\\

Consider the 3 point model of incidence geometry.  In this model (we have previously shown this to be a model), we have 3 points, call them A,B, and C and three lines, call them AB, BC, and AC.  Here is is clear that no matter which 2 lines we choose, all points in the model are contained in just those two lines.  Since this statement is not true in this model, where all of the axioms hold, it clearly cannot be a result of the axioms.\\



b. Show, however, that statement S holds in every projective plane.  Hence ~S cannot be proved from the axioms of incidence geometry either, so S is independent of those axioms.\\

Given a projective plane M.  From Dr. Frey's bonus but required fun question we know that we will end up with distinct points and lines that can be named in this manner.  Points: A,B,C,D,E,P; lines: ABC, PAD, PBE, and PC (although we are not limited to these points and lines necessarily). \\ 

We know that all of these points and lines must be present (though they may be differently named).  I will exhaustively show that any pair of lines will not contain all points already demonstrated to be in the model.\\

Given two lines. Since there are 6 points, suppose that a single line contains at least 3 of these points (if neither line can contain at least 3 points, then between both of them they clearly cannot contain all 6).  Since it contains at least 2 of the points, by I1 it must designate a unique line.  \\

Suppose it contains A.\\

Case 1: It also contains B or C.  Then by I1 we know this designates the line ABC and therefore it contains A,B, and C.  It does not contain P, D or C as any of these choices would result in ABC being equal to some other predefined line (If it contains D then it must be line PAD by I1, etc).  Also notice that if ABC is our first line, the other line cannot contain P,D and E as then, by I1, it PAD would be the same line as PBE. So neither of our lines can be the same as line ABC if we wish to contain all 6 mentioned points (and if we cannot do that, we obviously cannot contain all points in the model).\\

Case 2: It also contains either P or D.  Then it must be the line PAD by I1.  Additionally, it cannot contain B or C as this would result in PAD being the same as ABC, which would be a contradiction, and it cannot contain E as this would mean it contains P and E and by I1, PAD is the same as PBE as they both contain both P and E (contradiction).  So it cannot contain more than 3 points again.  Additionally, if it is PAD, the other line must contain B,C, and E.  However, this means that the line ABC and PBE are the same line (I1).  Contradiction.  Therefore the first line cannot be PAD if we wish to contain all 6 of our known points on just 2 lines. \\

Case 3: It also contains E. Then it composes some line we have not previously defined.  Additionally, it cannot contain any of the other known points (I1) as it would then be the same as either PAD (if it contains P or D) or ABC (if it contains B or C).  Therefore our other line must contain P, B, C and D.  However, by I1, we know that means our second line is PBE, PC, PAD and since it therefore contains A and C, also must be ABC.  However, these are all distinct lines. Therefore we cannot have a second line which contains all points in the model.\\

This means that if one of our lines contains A, any choice of two lines will not contain one of the 6 points known to be in all projective planes.  Additionally, if neither line contains A, then A is not incident with either line and again we do not contain every point in the model.\\

Therefore the statement is true since no matter the choice of two lines, there is always a point not incident with either of them.\\





c. Use statement S to prove that in a finite projective plane, all the lines have the same number of points lying on them.  Hint: Map the points on l onto points on m by projecting from the point P.  This mapping is called a perspectivity with center P.\\

First consider the point that is the intersection between m and l.  This point maps to itself.  This does not need to be explicitly stated but I find it helpful to state it.\\

I propose a 1-1 relation between the points on the two lines as follows.  Given a point A on line l, there is exactly 1 line connecting A and P (I1).  Due to the elliptic parallel property (and the fact that A and P are not on m), the line AP intersects m.  \\

Injective: Given 2 points on l, call them A and B, that map to the same point on m, call it A'.  We know by the definition of our mapping that A' and P are on the same line.  Therefore this line is unique (I1).  Therefore A and B are on this line.  Assume to the contrary that A and B are not the same point (as we would be done), so A'P intersects l at two point.  By I1, this means that l and A'P are the same line.  Therefore P lies on l.  Contradiction.  Therefore A and B are the same point.\\

Surjective: Given a point A' on m that is not the intersection of m and l, consider the line connecting A' and P.  Since we have the elliptic parallel property in this model, A'P must intersect l, say at A.  Note that A is therefore on the line APA'.  Consider the line connecting A and P.  By I1, this is the same line APA'.  Therefore, since we know that m is a distinct line (distinct from APA'), then we know that APA' only intersects m once, which is at A'.  Hence we have found a point on l that maps to the given point A'.\\



Therefore we have a bijective relationship between the points on the lines, so both lines have the same number of points. \\



d.  Prove that in a finite affine plane, all the lines have the same number of points lying on them.  Hint: Apply part (c) to the projective completion or find a direct affine proof.

Given a finite affine plane, we can take the projective completion of the affine plane and we know that the result in part c holds.  In order to create the projective completion of the affine plane, we had to add a number of ``points at infinity'' and a ``line at infinity'' to connect the points at infinity to satisfy I1.  We know by part c that this line at infinity has the same number of points as every other line.  Additionally, this line at infinity intersects every other line exactly once (and is incident with every added point at infinity). Therefore, assuming there are n+1 points on the line at infinity (we know this to be the case since we were working with a finite affine plane), we know that every line in the projective completion has n+1 points on it.  Therefore when we remove the line at infinity (returning to the original, given finite affine plane), we are removing 1 point from every line.  Therefore every line has n points on it.  Therefore every line has the same number of points. 





\prob{Major exercise 2}  Show that every projective plane P is isomorphic to the projective completion of some affine plane. 



Given some projective plane P.  Given some line m in P (we know there are at least 3 points by I3, and therefore there are lines), m is similar to all other lines in the projective plane in that they all are incident with the same number of points.  We will call m the line at infinity and remove it as well as every point incident with it.  Now we must show that what we are left with is an affine plane.  \\

\noindent Before removing the line:\\

Note that due to I2, the line at infinity had at least 3 points on it, call them A,B, and C.  Additionally, we know that by I3 there were 3 non-collinear points, assuming 2 of these are among A,B, and C, call the non-collinear point (non-collinear to A,B, and C) D.  Therefore we know there will be at least 1 point remaining in our model after we remove line ABC.  By I1 we know that there was some line connecting A and D.  By I2 (strong) we know that this line had 3 points on it.  By I1 we know that only 1 of these points could have been incident with line ABC since if more than 1 point was incident with ABC then line AD would be equal to line ABC (I1).  Therefore D (I3) would be collinear with ABC which is a contradiction.  Therefore there are at least 2 points remaining in the model, call the other one E.  By I1 they form the line ADE.  By I1 we know there was also a line BE.  By I2 (strong) we know that there is a third point on BE, call it F.  Suppose that BEF is the same as line ADE.  Then ABDEF are all points on the same line.  By I1, we know that ABC is therefore also the same line as both lines share points A and B.  Therefore A,B,C and D are collinear. Contradiction.  Therefore ADE and BEF are distinct lines, so D,E, and F are non-collinear.  \\

Now consider the model with the line ABC removed as well as removing all of the points on the line.  We must show the 3 incidence axioms hold as well as showing the Euclidean parallel property holds to show that this is an affine plane. \\

I1: Given two points in the orginal model, both not on the line ABC, we know that these two points form a unique line (I1 in original model).  Additionally, since our two points are not on ABC, this line is not equal to ABC.  Therefore this line is in the model after we remove ABC.\\

I2:  Given a line m not equal to the line ABC in the model after the removal of ABC (we have shown that there are still lines, such as DE and EF) we know that there were at least 3 points on the line before the removal of ABC.  Additionally, due to the elliptic parallel property, the line was incident with ABC at at least one point.  Suppose it was incident with more than one point.  By I1, m is the same line as ABC.  Contradiction.  Therefore m was incident with ABC at exactly one point.  With this point removed, we know that there are still 2 points on the line.  Therefore I2 is satisfied.\\

I3:  Since ADE and BEF were distinct lines, we know that D,E, and F are non-collinear. \\

Euclidean parallel property:  Given a line l and a point M in the model after ABC has been removed, suppose to the contrary that there are at least 2 lines, call them g and h, parallel to l through M.  Since g and h are parallel to l, they are not incident with it.  Since the original model has the elliptic parallel property, g and h were both incident with l at some point Q that has clearly been removed and is therefore on ABC.  Therefore h and g share points M and Q and by I1 (original model) they are the same line.  To show that they are still the same line in the new model (I suppose one could call this into question), we know that by I2 (strong) there are 3 points on g and h.  We know of M and Q already, so call the new point R.  If we suppose R is on ABC then by I1, since R and Q are both coincident with ABC, the line MQR is the line ABC.  Therefore M was removed from our model when ABC was removed from our model.  Contradiction.  Therefore M and R are still in our model after removing ABC, g and h are both the line MR (I1), so there is at most 1 line parallel to l through M.  \\

Assume then that there are no lines parallel to l through M.  Consider again the model with ABC in it.  We know that ABC is incident with l at 1 point, call it S, as they are distinct lines (I1) and as we have the elliptic parallel property.  Consider the line SP.  By I2 (strong) we know that there will be a third point on the line, call it T.  Suppose that point is incident with ABC, then by I1 MP is ABC.  However, M was in our model after the removal of ABC and the removal of all points incident with ABC.  Contradiction.  Therefore, by I1, there is a line MT in the model after the removal of line ABC.  Suppose that this line is incident with line l and note that this point of incidence cannot be on ABC as every point incident with ABC has been removed.  Then in the original model with ABC in it, the line l is incident with the line MP at two locations.  Therefore by I1, MP and l are the same line.  Therefore P is incident with l.  Contradiction.  Therefore MP is parallel to l.\\

Therefore since, given a line l and a point M not on l, there cannot be more than 1 parallel line through M and since there has to be at least one parallel line through M, the Euclidean parallel property holds.  Hence the model with any arbitrary line (in this case ABC) removed is an affine plane.\\



Now we must show that our projective plane P is isomorphic to the completion of the affine plane.\\

Note that we can create a 1-1 correspondence between points and lines with the following method.  If the point is in the affine plane, is not on the line ABC, however, it is indeed in the projective plane.  Map the point in the affine plane to the point it originated as in the projective plane.  If the point is in the completion of the affine plane but not in the affine plane itself, then we know that it is incident with lines that are parallel to each other in the affine plane.  However, we know that these parallel lines similarly meet at a point in the projective plane.  Map this point in the projective plane to the point we just mentioned in the completion of the affine plane.  So we know the mapping of points in the projective plane to the points in the completion of the affine plane is 1-1.\\

Similarly, a line in the affine plane has a corresponding line in the projective plane.  When we take the completion of the affine plane, we make a point to be incident with all lines that are parallel to each other in the affine plane and a ``line at infinity'' to connect all of these points to satisfy I1.  This ``line at infinity'' will map to the line we removed from the projective plane (ABC). We have shown that we have a 1-1 correspondence between the sets of lines.  \\

Now we must show that given a point U and a line o in the projective plane that U' and o' (the point and line they map to in the completion of the affine plane) that U is incident with o if and only if U' is incident with o'.  \\

$\Rightarrow:$ Given a point U and a line o in the projective plane such that U is incident with o.\\

Case 1: line o is the line ABC.  Therefore U is a point that is incident with lines that are parallel in the affine plane created by removing o.  Since this is the case we know that U' is a ``point at infinity'' in the completion of the affine plane.  Therefore, since o' is the line created by connecting all of the ``points at infinity'', we know that o' is incident with U'.\\


Case 2: line o is not the line ABC.\\

Subcase 1: point U is incident with line ABC. \\

Since U is incident with ABC we know U' is a point at infinity.  Also, on removing U all lines that are incident with U become parallel, so consider another line (not ABC) that meets at U and a point on that line, call it X.  We know that this other line through X is parallel to o in the affine plane.  By our mapping, we find U' by taking parallel lines in the affine plane which mapped to lines in the projective plane which met on ABC and finding the point in the completion where they intersect.  So U' is the point where o' and the line through X' meet.  Therefore U' is incident with U'.\\


Subcase 2: point U is not incident with line ABC.  In this subcase, we have a line and point that map directly into the affine plane (as opposed to the extension of it).  Since this is the case, we know that this point and line map to a point and line that are incident with each other as this portion of the projective plane is ``copied'' to create the affine plane (and our mapping is the natural mapping). \\


$\Leftarrow:$  Given a point U' and line o' in the completion of the affine plane such that U' is incident with o'.\\

Case 1:  o' is the line added when going from the affine plane to the completion.  Therefore we know that o is ABC by our mapping.  Since U' is incident with o' we know that it is incident with lines that were parallel in the affine plane, these lines in the affine plane are lines in the projective plane (not ABC) and this point maps to the point where these meet.  However, we know that these lines are parallel after the removal of ABC and all of the points on ABC.  Therefore U is on ABC, so U is incident with o.\\

Case 2: o' is a line in the affine plane (not in the completion)\\

So we know that o is in the projective plane but is not equal to ABC.  \\

Subcase 1: U' is a point at infinity.\\

Then we know that there are lines in the affine plane that are parallel to o'.  We know that these lines meet at U' in the completion of the affine plane. For o and a line parallel to o (in the projective plane with ABC removed) we know that U is the point on ABC where they meet in the projective plane (by our mapping).  Since o and the line parallel to o meet at U it is obvious that U is incident with o.\\

Subcase 2: U' is not a point at infinity.  Therefore we know that U is a point in the projective plane that is not incident with ABC.  Then we know that U' and o' are in the affine plane so they map naturally back to the point/line they were created from by removing ABC from the projective plane.  Because of this we know that U is incident with o.\\


Hence the projective plane is isomorphic to the completion of some affine plane.  










\prob{Major exercise 3} Let P be a finite projective plane so that, according to Exercise 14(c), all lines in P have the same number of points lying on them;  call this number n+1, with $n \geq 2$.  Show the following:\\

a. Each point in P has n+1 lines passing through it.  \\

Given a point Q in P, consider any line l not incident with Q (by I3 we know this is possible).  So we know that l has n+1 points on it.  Additionally, given any line m incident with Q, we know that m is incident with l (elliptic parallel property).  Finally, given a point on l we know there is exactly one line incident with this point and Q (I1).  Therefore there is exactly one line passing through Q for every point on l, and there are n+1 points on l, therefore n+1 lines pass through any given point in P.\\



b. The total number of points in P is $n^2 + n + 1$\\

Given a line l in P, there are n+1 points on l.  By the elliptic parallel property and I1, every line not equal to l is incident with l at exactly 1 point.  Additionally, each point on l has n+1 lines going through it, meaning there are n lines not equal to l that pass through each.  Consider a point Q on l. Given any point not on l, we know that there is some line that connects this point and Q (I1).  Additionally, by I1 we know that any line incident with l at Q is not incident with any other line incident with l at Q as they would not be distinct lines.  Therefore, if we count the points on lines incident with l at Q (not including Q) then we have counted every point in P except those on l.  At that stage we must add the n+1 points on l to our sum.  Therefore, in counting the points in P not on l, we know that each of the n lines (other than l) incident with Q have n+1 points on them.  Excluding Q from each there are $(n)(n) = n^2$ points in P not on l.  Finally, this means there are $n^2 + n + 1$ points in P.\\


c. The total number of lines in P is $n^2 + n + 1$.  The number n is called the order of the finite projective plane.\\

Given a line l in P, there are n+1 points on l.  By the elliptic parallel property and I1, every line not equal to l is incident with l at exactly 1 point.  Additionally, each point on l has n+1 lines going through it, meaning there are n lines not equal to l that pass through each.  \\

Assume to the contrary that the set of lines (other than l) going through one point on l and the set of lines (other than l) going through another point on l are not disjoint.  Then there is some line in both sets.  Since it is in both sets, it must be incident with at least 2 points on l.  Therefore, by I1, it is l.  Contradiction.  Therefore there are n distinct lines (other than l) going through each point on l.  So there are $(n+1)(n) = n^2 + n$ lines other than l that go through points on l.  Including l we have $n^2 + n+ 1$ lines in P.\\




\prob{14}  If a, b and c are rays, let us say that they are coterminal if they emanate from the same point, and let us use the notation a*b*c to mean that b is between a and c (as defined on p. 115).  The analogue of Axiom B-1 states that if a*b*c, then a,b,c are distinct and coterminal and c*b*a;  this analogue is obviously correct.  State the analogues of Axioms B-2 and B-3 and Proposition 3.3 and tell which parts of these analogues are correct.  (Beware of opposite rays!)\\

B-2: Given coterminal rays b and d, there exist rays a,c, and e coterminal to b and d such that a*b*c, b*c*d, and c*d*e.\\

Suppose b and d are opposite rays, then we cannot have a ray between them as this is not defined, so b*c*d may not be true.  \\

Since we say, ``there exists rays'' we know that a and c may not be opposite, which means a*b*c will be true.  Additionally, this means that c and e may not be opposite so we can say c*d*e.  This all of course assumes the existence of points that allow us to create these rays.  If we are allowed to do constructions, then we can create them. This is evident since we can bisect the angle created by b and d to find a, recreate that angle (less than 90 degrees as b and d are not quite opposite) on the other side of b to find a and on the other side of d to find e.  Note that both pairs $(a,c)$ and $(c,e)$ are then not quite 180 degrees from each other since they are double the angle between b and d which is less than 90, so they are not opposite.  Therefore we can modify our statement restricting b and d from being opposite to make the analogue true.\\




B-3: If a, b, and c are coterminal rays, then one and only one ray is between the other two.\\

This is false.  If any pair of rays are opposite, betweenness is not defined. If no pair of rays is opposite, then we can have rays 120 off from each other.  This would also allow no ray to be between the other two. \\ 

It is clear that we can have no rays between the other two, it is also the case that we can have up to one ray between the other two (with no other rays between the other two simultaneously).  This can be shown by considering rays that are almost opposite with a ray between them.  Note that if we go further than making them opposite we have the same problem reflected, so we only need to consider it once.  So we can consider the idea of making the two outer rays closer together while keeping the third ray between them.  In this case, we should try to move the interior ray so that it remains between the other two rays but also becomes the ``outside'' ray so that one of the other rays is between it and the third ray.  We cannot do this however, as long as it is between the other two rays we cannot force it outside while keeping it between the other rays.  Therefore we can have at most one betweenness relationship between 3 rays.\\



Proposition 3.3: Given 4 coterminal rays a,b,c, and d such that a*b*c and a*c*d, then b*c*d and a*b*d.  \\

From this we can deduce that a and c are not opposite rays and that a and d are not opposite rays (as you cannot be between opposite rays).  Additionally, since not being opposite rays means there is less than 180 degrees between them and since each other ray is between a and d, we know that no pair of rays mentionable among these rays are opposite. \\

Let the point each array emanates from be P and let the ray a have the point A, ray b have B etc.  Consider the line DP.  We know that since a and d are not opposite rays, that A is on some side of DP.  Since a*c*d we know that A and C are on the same side of D.  Also, since the interior designated by the rays a and c is a subset of the interior designated by rays a and d, B is on the same side of DP as well.  Similarly, B,C and D are on the same side of PA.  Therefore we know that a*b*d.  Also we know that since A and B are on the same side of PC (a*b*c) and since A and D are on opposite sides of PC (a*c*d), that C and D are on the same side of B. So we know that b*c*d.\\




\prob{15}  Find an interpretation in which the incidence axioms and the first two betweenness axioms hold but Axiom B-3 fails in the following way:  There exist three collinear points, no one of which is between the other two.  (Hint:  In the usual Euclidean model, introduce a new betweenness relation A*B*C to mean that B is the midpoint of AC.)\\

Consider a Euclidean plane and let betweenness mean that if A*B*C then B is the midpoint of segment AC.  Note that the incidence axioms do not depend on betweenness and therefore holds since this is a Euclidean plane.  We must show the first two betweenness axioms hold and that the third does not.\\

B1:  Given that A*B*C, we know that to be a midpoint of segment AC, you have to first have a segment which means that A is not equal to C.  Additionally, once you have a segment, the midpoint is not equal to either of the endpoints.  Therefore A, B, and C are distinct points.  Additionally, AC and CA are the same segment and B is the midpoint either way, so C*B*A.\\

B2:  Given 2 distinct points B and D, we know by I1 that there is a unique line BD.  Additionally, there is a segment BD.  We can find the midpoint of this segment and name it C so B*C*D.  Additionally, we can take the segment BC and lay it off starting at B in the direction opposite C and call the new endpoint A (Congruence axiom 1: there is a unique point on the ray starting at B which does not include C such that $BA \equiv BC$).  Similarly we can lay BC off CD off starting at D in the direction opposite C calling the new endpoint E.  Note that since AB is congruent to BC, B is the midpoint of AC so A*B*C.  Similarly, C*D*E.\\

~B3: Consider A, B, and D from the proof for B2.  A,B, and D are on the same line, however, C and D are distinct points and B is the midpoint of AC. So B is not the midpoint of segment AD.  Additionally, D is not on the segment AB, so it is not the midpoint of AB.  Similarly, A is not on BD, so A is not the midpoint of BD.  Therefore none of the following statements is true.  A*B*D (or its alias D*B*A), A*D*B, B*A*D.\\








\prob{16} Find an interpretation in which the incidence axioms and the first three betweenness axioms hold but the line separation property (Prop 3.4) fails.  (Hint:  In the usual Euclidean model, pick a point P that is between A and B in the usual Euclidean sense and specify that A will now be considered to be between P and B.  Leave all other betweenness relations among points alone.  Show that P lies neither on ray AB nor on its opposite ray AC.)\\

Consider a Euclidean plane.  Since we are in a Euclidean plane, we can choose a particular line, call it l, and three distinct points on that line, call them A,P, and B such that A*P*B.  For this one set of points, redefine it so that P*A*B and B*A*P, but let betweenness remain the same for every other triplet of points.\\ 

Note that the incidence axioms hold because they do not depend on the definition of betweenness and because they are already satisfied in the Euclidean plane.  We must show that the betweenness axioms hold (the first 3) but that Proposition 3.4 fails.  More specifically, we only need to show that the first 3 betweenness axioms hold when our triplet of points is $\{A,B,P\}$.  In every other case we already know it to be true.\\



B1: Given a set of points X,Y,Z such that X*Y*Z, either X,Y, and Z are the points P, A, and B or they are not.  If they are we already know that they are distinct points and that both P*A*B and B*A*P.  If X,Y, and Z are not P,A, and B, then we know that the axiom is satisfied since it is usually satisfied with these three points in the Euclidean plane. \\

B2: Given a set of 2 points, we know that there is an infinite number of points between them and on either side that would allow us to satisfy the axiom.  There is a catch though.  We can run into complications if we at some point include A,B, and P in the set of 5 points simultaneously.  If we avoid this (and the infinite number of points mentioned earlier this paragraph means that we can) then we don't have to do any special case work since this axiom is satisfied for all pairs of points that can be brought forward.  The only thing that needs to be done is that when we are choosing our 3 points, we must always choose points other than A,B, and P. \\

B3: Given a set of three points.\\

Case 1:  We have the set A,B, and P.  We know that for this triplet it is the case that P*A*B and B*A*P.  That is, only A can be between the other two. Hence the axiom is satisfied in this case.\\

Case 2:  We have some set that has some point other than A,B, and P (at least one point is different than case 1).  This triple satisfies this axiom in the normal Euclidean plane and betweenness is defined the same for this triplet as it is for the normal Euclidean plane.  Therefore the axiom is satisfied in this case.\\


Proposition 3.4: Assume proposition 3.4 holds and note that P*A*B, however, we know that some point C exists such that C*A*P (B2). This would mean that C is incident with ray AP or AB, however, it is not.  Contradiction.  Therefore proposition 3.4 does not hold.\\





\prob{17}  A rational number of the form $\frac{a}{2^n}$ (with a,n integers) is called dyadic.  In the interpretation of Example 1 (p. 117) for this chapter, restrict to those points which have dyadic coordinates and to those lines which pass through several dyadic points.  The incidence axioms, the first three betweenness axioms, and the line separation property all hold in this dyadic rational plane; show that Pasch's theorem fails.  (Hint:  The lines 3x + y = 1 and y = 0 do not meet in this plane.)\\


Consider the points at (1,4), (0,0), and (1,0) and notice that they are all dyadic points.  Also notice that the line y = 5x connects (1,4) and (0,0) (both dyadic points).  Also, the line y = 1-3x intersects this line at ($\frac{5}{8}$,$\frac{1}{8}$) which is a dyadic point.  Additionally, the line y=1-3x intersects y=0 at $\frac{1}{3}$ which is not a dyadic point.  Since, in the Euclidean plane we know that these lines intersect at only 1 point,  and since this point is not dyadic, we know that these lines do not intersect in our dyadic plane.  Additionally, the line x=4 connects the points (1,0) and (1,4) but this line intersects y=1-3x at the point (1,-2) which is not between the other two points.  Therefore Pasch's theorem fails as we have found three points and their corresponding lines, and another line not incident with those points that intersects one of those lines between the points, but does not intersect either the second or third line between their points.  \\




\end{document}