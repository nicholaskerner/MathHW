%Latex2e file
\documentclass[12pt,letterpaper]{article}
%\renewcommand{\arraystretch}{2}
%\input{\scrload.tex}
\setlength{\textwidth}{6.5in}
\setlength{\textheight}{9.5in}

\setlength{\oddsidemargin}{-.25in}
\setlength{\evensidemargin}{-.25in}
\setlength{\topmargin}{-.25in}
\pagestyle{empty}

\usepackage{amsmath}
\usepackage{amssymb}
\usepackage{graphicx}

\newcommand{\R}{\ensuremath{{\mathbb{R}}}}
\newcommand{\Z}{\ensuremath{{\mathbb{Z}}}}
\newcommand{\Q}{\ensuremath{{\mathbb{Q}}}}
\newcommand{\N}{\ensuremath{{\mathbb{N}}}}
\newcommand{\C}{\ensuremath{{\mathbb{C}}}}
\newcommand{\Proof}{\noindent {\bf Proof: }}
\newcommand{\QED}{\begin{flushright}QED\end{flushright}}
\newcommand{\Refl}{{\bf Reflexive: }}
\newcommand{\Symm}{{\bf Symmetric: }}
\newcommand{\Tran}{{\bf Transitive: }}
\newcommand{\ep}{\varepsilon}
\newcommand{\ri}{\right|}
\newcommand{\lef}{\left|}
\newcommand{\toR}{\to \R}
\newcommand{\fancy}[1]{#1_{\text{fancy}}}
\newcommand{\pro}[1]{\noindent {\bf #1}}
\newcommand{\prob}[1]{\newpage\noindent {\bf #1}}
\newcommand{\bacon}{\approx}

   
\begin{document}
\begin{flushright}
Nick Kerner

Homework 4

Chapter 3: 24, 27, 28, 32, 35\\
Chapter 4: 5 (first part only), 17
\end{flushright}
\begin{center}
\large{Geometry}\\
\end{center}

\pro{24} Justify each step in the following proof of Proposition 3.14.

Given $\angle ABC \cong \angle DEF$.  To prove $\angle CBG \cong \angle FEH$:\\


\noindent (1) The points $A,C$ and $G$ being given arbitrarily on the sides of $\angle ABC$ and the supplement $\angle CBG$ of $\angle ABC$, we can choose points $D,F,$ and $H$ on the sides of the other angle and its supplements so that $AB \cong DE$, $CB \cong FE$, and $BG \cong EH$.\\

First, we know we can extend the line $AB$ so that we can have the supplementary angle $\angle CBG$.  Next, we know we have rays $\overrightarrow{BA},\overrightarrow{BC},\overrightarrow{BG}, \overrightarrow{ED}, \overrightarrow{EF},$ and $\overrightarrow{EH}$.  Now since we have points $A,C,$ and $G$, we know we have segments $AB,CB,$ and $GB$.  By congruence axiom 1, there is a point $D$ on $\overrightarrow{ED}$ such that $DE\cong AB$.  Similarly, by C1 we can find $F$ and $H$ such that $CB \cong FE$ and $BG \cong EH$.\\

\noindent (2) $\triangle ABC \cong \triangle DEF$.\\

We know that $\triangle ABC \cong \triangle DEF$ since $AB \cong DE$, $CB \cong FE,$ and $\angle ABC \cong \angle DEF$ (SAS).\\

\noindent (3)Hence $AC \cong DF$ and $\angle A \cong \angle D$.\\  

Definition of congruent triangles, page 119.\\

\noindent (4) $AG \cong DH$\\

We know that $AB \cong DE$ and $BG \cong EH$ from above.  Second we know that since $\overrightarrow{BA}$ and $\overrightarrow{BG}$ are opposite rays and that $\overrightarrow{ED}$ and $\overrightarrow{EH}$ are opposite rays, that $A*B*G$ and $D*E*H$.  So by C3 we know $AG \cong DH$.\\

\noindent (5) $\triangle ACG \cong DFH$. \\

We know that $\angle A \cong \angle D$, $AG \cong DH$, and $AC \cong DF$ from above.  By SAS (C5) the triangles are congruent.\\

\noindent (6) $CG \cong FH$ and $\angle G \cong \angle H$\\

Definition of congruent triangles.\\

\noindent (7) $\triangle CBG \cong \triangle FEH$\\

We know that $CG \cong FH$, $\angle G \cong \angle H$, and $BG \cong EH$ from above.  Therefore by SAS, the triangles are congruent.\\

\noindent (8) $\angle CBG \cong \angle FEH$\\
 
Definition of congruent triangles.

 

\prob{27}Prove Prop 3.18.  

\Proof

Given $\triangle ABC$ in which $\angle B \cong \angle C$.  By C2 we know that $BC$ is congruent to itself.  By ASA (Prop 3.17) we know that $\triangle ABC \cong \triangle ACB$.  Therefore by the definition of congruent triangles, corresponding segments AB and AC are congruent.  Also since $AB \cong AC$, we know that $\triangle ABC$ is isosceles. 

\QED




\prob{28}Prove that an equiangular triangle is equilateral.

\Proof

Given an equiangular triangle $\triangle ABC$.  By definition $\angle A \cong \angle B$, $\angle B \cong \angle C$, and $\angle A \cong \angle C$.  By proposition 3.18 we know that since $\angle B \cong \angle C$, that $AB \cong AC$.  Similarly since $\angle A \cong \angle B$, we know that $AC \cong BC$.  By (C2) we know that $AB \cong BC$, so we have an equilateral triangle.

\QED

\prob{32} Prove Proposition 3.22.


\Proof

Given triangles $\triangle ABC$ and $\triangle DEF$ such that $AB \cong DE$, $BC \cong EF$, and $AC \cong DF$.  

By C1 we know that we can lay a copy of DE off on the ray $\overrightarrow{AB}$ starting at the point A, so A = D'.  Additionally, we know that if we lay it off starting at A, there is a unique point, say E', such that $AE' \cong DE$.  However we know that $DE \cong AB$, so $E' = B$.  

By the corollary to SAS, we know that there exists two points (one on either side of $\overleftrightarrow{AB}$), lets call it F' such that $\triangle ABF' \cong \triangle DEF$.  Additionally choose the point such that F' and C are on opposite sides of $\overleftrightarrow{AB}$.  Let $P$ be the point where $F'C$ intersects $\overleftrightarrow{AB}$ (definition of opposite sides of a line).  

Case 1: P *B*A

Since $P \neq B$ we know that $F', C, $ and $P$ form $\triangle F'BC$.  Since $\triangle ABF' \cong DEF$ we know that $F'B \cong FE$ and we know that $FE \cong CB$ so by C2 we know that $F'B \cong CB$.  So by Proposition 3.10, $\angle BF'C \cong \angle BCF'$.

Additionally, we have a big triangle, $\triangle ACF'$ and we know that $AC \cong DF$ and $DF \cong AF'$ ($\triangle ABF' \cong DEF$)so $AC \cong AF'$ (C2).  Therefore by Proposition 3.10 we know that $\angle ACF' \cong \angle AF'C$. So by Proposition 3.20, $\angle AF'B \cong \angle ACB$ and by C5 since $\angle AF'B \cong \angle DFE$ we know that $\angle ACB \cong \angle DFE$.  Therefore since $AC \cong DF$ (by assumption), $BC \cong EF$ (by assumption), and $\angle ACB \cong \angle DFE$ we know that $\triangle ABC \cong \triangle DEF$ by SAS.\\

Case 2:  P= B

Consider the triangle $\triangle ACF'$.  We know that $AC \cong DF$ and $DF \cong AF'$ (since $\triangle AF'B \cong \triangle DFE$) so $AC \cong AF'$ (C2). Therefore by Proposition 3.10 we know that $\angle ACF' \cong \angle AF'C$.  Also note that $\angle ACF'$ is an alias for $\angle ACB$ since $P=B$ and since $P$ is on $\overrightarrow{CF'}$.  Similarly, $\angle AF'C$ is an alias for $\angle AF'B$, so $\angle ACB \cong \angle AF'B$.  Since $\triangle AF'B \cong \triangle DFE$ we know that $\angle AF'B \cong \angle DFE$, so by C5 we know that $\angle ACB \cong DFE$.  Also since we already know that $AC \cong DF$ and $CB \cong FE$ (by assumption) we know that $\triangle ABC \cong \triangle DEF$ by SAS.\\


Case 3: A*P*B

Consider the triangle $\triangle ACF'$. We know that $AC \cong DF$ and $DF \cong AF'$ (since $\triangle AF'B \cong \triangle DFE$) so $AC \cong AF'$ (C2).  Therefore by Proposition 3.10 we know that $\angle ACF' \cong \angle AF'C$. 

Similarly in $\triangle BCF'$ we know that $CB \cong FE$ and $FE \cong F'B$ (since $\triangle AF'B \cong \triangle DFE$) so $CB \cong F'B$ (C2).  Therefore by Proposition 3.10 we know that $\angle BCF' \cong \angle BF'C$. 

Therefore by Proposition 3.19, we know that $\angle ACB \cong \angle AF'B$.  Also since $\triangle AF'B \cong \triangle DFE$ we know that $\angle AF'B \cong \angle DFE$.  Therefore by C5 we know that $\angle ACB \cong DFE$.  Also we know (by assumption) that $AC \cong DF$ and $CB \cong FE$, so by SAS we know that $\triangle ABC \cong \triangle DEF$.\\


\noindent Case 4: P = B

The argument is effectively the same as case 2.\\

\noindent Case 5: A*B*P

This case is effectively the same as case 1.

\noindent Hence $\triangle ABC \cong \triangle DEF$.

\QED





\prob{35}



C1: Given some segment (or more importantly, the length of that segment) and some ray, we can find a segment along this ray such that the two segments are congruent. So we know this axiom holds

C2: Given segments $AB \cong CD$, $CD \cong EF$, we know that AB and CD have the same numerical length and so does EF.  Therefore since EF and AB have the same length, they are congruent.  Also, every segment has the same length as itself/is congruent to itself.

C3: If A*B*C and A'*B'*C' such that $AB \cong A'B'$ and $BC \cong  B'C'$, then we know that AB and BC are collinear and we also know that A'B' and B'C' are collinear.  Therefore we know that the distance $AC = AB + BC = A'B' + B'C' = A'C'$, so $AC \cong A'C'$.

C4 and C5 are about angle congruence and are unaffected by the changes we have made as those are strictly about segment congruence.

Proposition 3.19:
The direction of a ray, or its angle or what angles it is congruent to is not effected by the change in distance measurement, so proposition 3.19 holds.


SAS counterexample:  Consider the equilateral triangle ABC, of which only 1 side is on the x axis, let that be AB.  Suppose SAS holds.  Then proposition 3.18 holds (as I based this off of SAS).  So we know that all 3 sides of ABC are congruent.  In the Euclidean plane we know then that this is an isosceles triangle, as the side on the x axis is double the length (and therefore not congruent) of the other sides.  Therefore we know that in the Euclidean plane that $\angle A \cong \angle B$.  However we know that the angles in our new plane are the same as the angles in our old plane.  Assume to the contrary that SAS holds.  Then construct a triangle DEF such that $\angle F \cong \angle C$, $EF \cong BC$, and $AC \cong DF$ and such that not side of DEF touches the x axis.  We know that $ABC \cong DEF$ in our new plane by SAS, since angles have not changed in the transition to the new plane.  However, we know that $DE\cong AB$ in the Euclidean plane, even if we use the length of the segments to determine congruence.  However, in this new plane, AB is twice as long as it was and is no longer congruent to DE, whose length has not changed.  This means that the triangles are not congruent.  Contradiction.  Therefore the SAS congruence axiom does not hold.

Therefo


\prob{5 (first part only)} Justify each of the 18 steps on page 167 proving that each segment has a midpoint.  (Proposition 4.3)

1. Let C be any point not on $\overleftrightarrow{AB}$.

I3 tells us that there are 3 noncollinear points.  Since A and B are in a line, there must be a point not in that line.  Call it C.\\


2. There is a unique ray $\overrightarrow{BX}$ on the opposite side of $\overleftrightarrow{AB}$ from C such that $\angle CAB \cong \angle ABX$.

C4 \\


3. There is a unique point D on $\overrightarrow{BX}$ such that $AC \cong BD$.  

C1 \\


4. D is on the opposite side of $\overleftrightarrow{AB}$ from C.

$\overrightarrow{BX}$ is on the opposite side of $\overleftrightarrow{AB}$ from C, D is an element of $\overrightarrow{BX}$, so either B is on the other side from C or $B\in \overleftrightarrow{AB}$.  

Assume to the contrary that $B=D$. So since $BD \cong AC$, $AC$ is only a point which means that $A,B,$ and $C$ are co-linear.  Contradiction. Since $B\neq D$, $B\not\in \overleftrightarrow{AB}$, so $B$ is on the other side from C\\


5. Let E be the point at which segment CD intersects $\overleftrightarrow{AB}$

Definition of "on opposite sides".\\


6. Assume E is not between A and B.  

If it were, we would have found a midpoint and we would be done.\\


7. Then either E=A or E=B, or E*A*B, or A*B*E.  

It is possible that $A=E$ or $E=B$ without additional justification.

As by 6. we know that A*E*B is not the case, So by B3 we know that either B is between the other two or A is. Hence the statement.\\


8. $\overleftrightarrow{AC}$ is parallel to $\overleftrightarrow{BD}$

Alternate interior angle theorem.\\


9. Hence, $E \neq A$ and $E \neq B$.  

$\overleftrightarrow{AC}$ is parallel to $\overleftrightarrow{BD}$, so either they do not intersect or they are the same line.  Assume to the contrary that they are the same line.  Then another alias for the line is $\overleftrightarrow{CD}$ which we know to contain E.  However, since A,B, and C are non-collinear, $\overleftrightarrow{CD} \neq \overleftrightarrow{AB}$.  However, $A\in \overleftrightarrow{CD}$ and $B\in \overleftrightarrow{CD}$.  So either $\overleftrightarrow{CD}$ and $\overleftrightarrow{AB}$ are the same line or A=B=E.  Since A,B, and C are non-collinear, so A=B=E.  However, by the problem statement we know that AB is a segment, so A and B are distinct.  Contradiction.  Therefore $\overleftrightarrow{AC}$ does not intersect $\overleftrightarrow{BD}$.

Without loss of generality, assume A=E.  Then $E \in \overleftrightarrow{CD}$ and $E\in \overleftrightarrow{AC}$.  Therefore both lines contain A and C, so by I1 they are the same line.  However, both $\overleftrightarrow{CD}$ and $\overleftrightarrow{BD}$ contain the point D, so they intersect.  Contradiction.  Therefore $E \neq A$ and $E\neq B$. \\

 


10. Assume E*A*B

Suppose A*B*E, then we know that A and E are on opposite sides of $\overleftrightarrow{BD}$, however, we also know that $C*E*D$, so C and E are on the same side of $\overleftrightarrow{BD}$.  However we also know that since C and D are on opposite sides of $\overleftrightarrow{AB}$ that $AC$ does not intersect $\overleftrightarrow{BD}$, so A and C are on the same side of $\overleftrightarrow{BD}$. By B4 we know that A and E are therefore on the same side of $\overleftrightarrow{BD}$.  Contradiction. Therefore we can assume E*A*B.\\





11 Since $\overleftrightarrow{CA}$ intersects side EB of $\triangle EBD$ at a point between E and B, it must also intersect either ED or BD.  

We know that $\overleftrightarrow{CA}$ intersects side EB of $\triangle EBD$ because $E*A*B$ and $A\in \overleftrightarrow{CA}$.  The rest is handled by Pasch's theorem.\\


12.  Yet this is impossible.

We know that $A\neq B$ and $A\neq E$, so they do not intersect at B or E.  Additionally, given a point $P\neq A$ on CA, we know that CP does not intersect $\overleftrightarrow{AB}$, so $P$ and $A$ are on the same side of $\overleftrightarrow{AB}$.  Similarly we know that D and every point (except B or E) on the segments ED and BD are on the same side of $\overleftrightarrow{AB}$ as the point D, which is on the opposite side of $\overleftrightarrow{AB}$ as C.  Therefore by B4, the segment CA cannot intersect BD or ED.



13 Hence A is not between E and B. 

Contradiction due to Pasch's theorem.\\


14.  Similarly, B is not between A and E.

I showed this in step 10, it appears I could have stated that we would lose no generality instead.\\


15.  Thus A*E*B

All other options lead to contradictions.\\

16.  Then $\angle AEC \cong \angle BED$

Alternate interior angle theorem ($E\in \overleftrightarrow{CD}$).

17. $\triangle EAC = \triangle EBD.$ 

SAA Congruence Criterion.

18.  Therefore E is a midpoint of AB.

By 17 we know that $AE \cong BE$. 
Definition of midpoint.





\prob{17}Prove the following theorems:

a. Let $\gamma$ by a circle with center O and let A and B be two points on $\gamma$.   The segment AB is called a chord of $\gamma$;  let M be its midpoint.  If $O\neq M$, then $\overleftrightarrow{OM}$ is perpendicular to $\overleftrightarrow{AB}$. Hint:  Corresponding angles of congruent triangles are congruent.

Assume $O \neq M$.

Notice that OA and OB are both radii of the circle. Because of this we know that they are congruent. Then the triangle OAB is isosceles, so we know that $\angle OAM = \angle OAB \cong \angle OBA = \angle OBM$ by proposition 3.10.

Since M is the midpoint of AB, we know that $AM \cong BM$.  Additionally, $OM \cong OM$.  Therefore since these two things are true and since $\angle OAM \cong \angle OBM$, we know that $\triangle OAM \cong \triangle OBM$.  Therefore $\angle OMA \cong OMB$ since they are corresponding angles in congruent triangles.  Therefore, since these angles are supplementary and congruent, they are right angles, so AB is perpendicular to OM.



b. Let AB be a chord of the circle $\gamma$ having center O. Prove that the perpendicular bisector of AB passes through the center O of $\gamma$. 

Name the point at which AB is incident with its perpendicular bisector M. Again we know that $AO \cong BO$ and $MO \cong MO$.  Also, again by Proposition 3.10 we know that $\angle OAM \cong OBM$.  Therefore by SAA we know that $AMO \cong BMO$.  Therefore $\angle AMO \cong \angle BMO$. Therefore $\angle AMO$ and $\angle BMO$ are right angles (as they are supplementary and congruent), so since $MA \cong MB$ and since $OM$ is perpendicular to AB, we know that OM is a perpendicular bisector of AB.  Additionally, by proposition 4.4 we know that the perpendicular bisector of a segment is unique.  Therefore the perpendicular bisector of a chord goes through the center of the circle.




\end{document}