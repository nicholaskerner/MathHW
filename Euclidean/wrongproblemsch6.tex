%Latex2e file
\documentclass[12pt,letterpaper]{article}
%\renewcommand{\arraystretch}{2}
%\input{\scrload.tex}
\setlength{\textwidth}{6.5in}
\setlength{\textheight}{9.5in}

\setlength{\oddsidemargin}{-.25in}
\setlength{\evensidemargin}{-.25in}
\setlength{\topmargin}{-.25in}
\pagestyle{empty}

\usepackage{amsmath}
\usepackage{amssymb}
\usepackage{graphicx}

\newcommand{\R}{\ensuremath{{\mathbb{R}}}}
\newcommand{\Z}{\ensuremath{{\mathbb{Z}}}}
\newcommand{\Q}{\ensuremath{{\mathbb{Q}}}}
\newcommand{\N}{\ensuremath{{\mathbb{N}}}}
\newcommand{\C}{\ensuremath{{\mathbb{C}}}}
\newcommand{\Proof}{\noindent {\bf Proof: }}
\newcommand{\QED}{\begin{flushright}QED\end{flushright}}
\newcommand{\Refl}{{\bf Reflexive: }}
\newcommand{\Symm}{{\bf Symmetric: }}
\newcommand{\Tran}{{\bf Transitive: }}
\newcommand{\ep}{\varepsilon}
\newcommand{\ri}{\right|}
\newcommand{\lef}{\left|}
\newcommand{\toR}{\to \R}
\newcommand{\fancy}[1]{#1_{\text{fancy}}}
\newcommand{\pro}[1]{\noindent {\bf #1}}
\newcommand{\prob}[1]{\newpage\noindent {\bf #1}}
\newcommand{\bacon}{\approx}

   
\begin{document}
\begin{flushright}
Nick Kerner

Homework 6

Chapter 5: 1, 2, 6, 16

\end{flushright}
\begin{center}
\large{Geometry}\\
\end{center}

\pro{1}Make a long list of geometric statements that are equivalent to the Euclidean parallel postulate.

The first 10 provided: \\


\prob{2} This problem has five parts.  In the first part we will construct Saccheri quadrilaterals associated with any triangle $\triangle ABC$.  Then we will apply this construction.  Figure 6.15 illustrates the case where the angles of the triangle at A and B are acute;  you are invited to draw the figure when one of these angles is obtuse or right.  

a. Let I,J,K be the midpoints of BC, CA, AB, respectively.  Let D, E, F, be the feet of the perpendiculars from A,B,C respectively, to $\overleftrightarrow{IJ}$ (which is called a medial line).  Prove, in any Hilbert plane, that $AD \cong CF \cong BE$, hence that $\square EDAB$ is a Saccheri quadrilateral with base ED summit AB.  Show that a triangle and its associated Saccheri quadrilateral have equal content-- ie, that you can dissect the Saccheri quadrilateral region into polygonal pieces and then reassemble these pieces to construct the triangular region. 

b. Prove that the perpendicular bisector of AB is also perpendicular to $\overleftrightarrow{IJ}$. (Hint: Use a result about Saccheri quadrilaterals.)  Hence if hte plane is hyperbolic, $\overleftrightarrow{IJ}$ is divergently parallel to $\overleftrightarrow{AB}$.  Assume now the plane is real, so lengths can be assigned (Theorem 4.3) and the Saccheri-Legendre theorem applies. 

c. Prove that $\overline{ED} = 2\overline{IJ}$.  Deduce that $\overline{AB} > 2\overline{IJ}$ (respectively $\overline{AB} = 2\overlien{IJ}$) if the plane is hyperbolic (respectively is Euclidean).

d. Prove that K,F, and C are collinear if and only if $AC \cong BC$ (isosceles triangle).  If that is the case, prove that F is the midpoint of CK iff the plane is Euclidean.  If K,F, and C are not collinear and the plane is not Euclidean, prove that F is the midpoint of CK iff the plane is Euclidean.  If K, F, and C are not collinear and the plane is not Euclidean, prove that $\overline{CF}$ is not perpendicular to $\overline{AB}$ (ray $\overrightarrow{CF}$ does intersect AB at some point G in the case shown, where the angles at A and B are acute, by the crossbar theorem, but CG is not an altitude of the triangle ifthe plane is not Euclidean).

e. Show that if the Pythagorean equation holds for all right triangles and if $\angle C$ is a right angle, then $\overline{AB} = 2\overlineIJ}$ can be proved.  Deduce from part c that such a plane must be Euclidean.  (Use these results to add to your answers in Exercise 1.)




\prob{6}

Let $\overrightarrow{PY}$ be a limiting parallel ray to  through P and let X be a point on this ray between P and Y (Figure 6.17).  It may seem intuitively obvious that $\overrightarrow{XY}$ is a limiting parallel ray to l through X, but this requires proof.  Justify the steps that have not been justified. 

1. We must prove that any ray $\overrightarrow{XS}$ between $\overrightarrow{XY}$ and $\overrightarrow{XY}$ meets l, where R is the foot of the perpendicular from X to l. 

2. S and Y are on the same side of $\overrightarrow{XR}$.  

3. P and Y are on opposite sides of $\overleftrightarrow{XR}$. 

4. By Exercise 5, S and Y are on the same side of $\overleftrightarrow{PQ}$.  

5. S and R are on the same side of $\overleftrightarrow{XY} = \overleftrightarrow{PR}.$

6. Q and R are on the same side of $\overleftrightarrow{PY}$.

7. Q and S are on the same side of $\overleftrightarrow{PY}$.

8. Thus, $\overrightarrow{PS}$ lies between $\overrightarrow{PY} $ and $\overrightarrow{PQ}$, so it intersects l in a point T.  

9. Point X is exterior to $\triangle PQT$.  

10. $\overrightarrow{XS}$ does not intersect PQ.  

11. Hence $\overrightarrow{XS}$ intersects QT (proposition 3.9a), so $\overrightarrow{XS}$ meets l. 




\prob{16}



\end{document}