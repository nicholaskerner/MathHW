%Latex2e file
\documentclass[12pt,letterpaper]{article}
%\renewcommand{\arraystretch}{2}
%\input{\scrload.tex}
\setlength{\textwidth}{6.5in}
\setlength{\textheight}{8.6in}

\setlength{\oddsidemargin}{-.25in}
\setlength{\evensidemargin}{-.25in}
\setlength{\topmargin}{-.25in}
\pagestyle{empty}

\usepackage{amsmath}
\usepackage{amssymb}
\usepackage{graphicx}

\newcommand{\R}{\ensuremath{{\mathbb{R}}}}
\newcommand{\Z}{\ensuremath{{\mathbb{Z}}}}
\newcommand{\Q}{\ensuremath{{\mathbb{Q}}}}
\newcommand{\N}{\ensuremath{{\mathbb{N}}}}
\newcommand{\C}{\ensuremath{{\mathbb{C}}}}
\newcommand{\Proof}{\noindent {\bf Proof: }}
\newcommand{\QED}{\begin{flushright}QED\end{flushright}}
\newcommand{\Refl}{{\bf Reflexive: }}
\newcommand{\Symm}{{\bf Symmetric: }}
\newcommand{\Tran}{{\bf Transitive: }}
\newcommand{\ep}{\varepsilon}
\newcommand{\ri}{\right|}
\newcommand{\lef}{\left|}
\newcommand{\toR}{\to \R}
\newcommand{\fancy}[1]{#1_{\text{fancy}}}
\newcommand{\pro}[1]{\noindent {\bf #1}}
\newcommand{\prob}[1]{\newpage\noindent {\bf #1}}
\newcommand{\bacon}{\approx}

   
\begin{document}
\begin{flushright}
Nick Kerner

Homework 9

Chapter 7: K11, K20

\end{flushright}
\begin{center}
\large{Geometry}\\
\end{center}

\pro{Chapter 7: K11 } Let $\Omega$ and $\Omega '$ be distinct ideal points and $A$ an ordinary point. Let $P$ be the pole of the chord $\Omega\Omega '$ and let Euclidean ray $\overrightarrow{AP}$ cut $\gamma$ at $\Sigma$. Prove that $A\Sigma$ represents the bisector of $\an\Omega A \Omega '$ in the Klein model. Apply this result to justify the construction of the line of enclosure given in Major Exercise 8, Chapter 6. (Hint: Use Proposition 6.6.) \\



\prob{Chapter 7: K20 } In any Hilbert plane, let $\square ABCD$ have both pairs of opposite sides congruent. Prove (1) that both pairs of opposite angles are congruent and that the lines containing opposite sides have a common perpendicular (use Exercise 12, Chapter 6), in particular are parallel. Such a quadrilateral will be called a \emph{symmetric parallelogram}. Prove (2) that $\square ABCD$ is a symmetric parallelogram iff the diagonals bisect each other. Let $S$ be their common midpoint. Prove (3) that the diagonals of a symmetric parallelogram are perpendicular iff all four sides are congruent, and in that case $\square ABCD$ has an inscribed circle with center $S$. Prove (4) that the diagonals of a symmetric parallelogram are congruent iff all four angles are congruent (in a semi-Euclidean plane, that happens iff $\square ABCD$ is a rectangle), and in that case $\square ABCD$ has a circumscribed circle with center $S$. $\square ABCD$ is called a \emph{regular 4-gon} if all four sides and all four angles are congruent. \\


\noindent b Prove that in a Euclidean plane, every parallelogram is symmetric, whereas in a hyperbolic plane, there exist parallelograms that are not symmetric. \\




\noindent c. Suppose that $\square ABCD$ is a symmetric parallelogram in a hyperbolic plane, with $S$ the midpoint of its diagonals. Show that for each pair of opposite sides, $S$ is the symmetry point for the lines containing those sides, in the sense of Major Exercise 12, Chapter 6. In the Klein model, suppose also that $S=O$. Show that $\square ABCD$ is a Euclidean parallelogram, that it is a Euclidean rectangle iff all four angles are Klein-congruent, and that it is a Euclidean square iff it is a hyperbolic regular 4-gon. \\

\noindent d.  In a Euclidean plane, use the results above about parallelograms to give a synthetic proof of the following theorem: In $\ti ABC$, let $B'$, $C'$ be the midpoints of $AC$, $AB$, respectively. $BB'$ and $CC'$ meet at a point $G$, by the crossbar theorem. Let $L$, $M$ be the midpoints of $BG$, $CG$, respectively. Then $BL \cong LG \cong GB'$ and $CM \cong MG \cong GC'$. In words: $G$ is two-thirds of the distance from each vertex to the opposite midpoint. Deduce from this that the three medians of $\ti ABC$ are concurrent. (Hint: Show that $\square LMB'C'$ is a parallelogram.) $G$ is called the \emph{centroid} of $\ti ABC$. \\




\end{document}